\begin{comment}
We see that the population carrying capacity, K and the terms from the density dependent mortality function $\gamma$ and $\theta$ all appear to be important across all four metrics. The harvest rate $\eta_{hunt}$ and the mean latency period $\sigma_1 mean$ also appear to be relevant in prevalence estimates and fadeout time respectively. These results seem reasonable, as we would expect increased hunting to provide better prevalence estimates. Similarly, the relationship between the latency period and time to fadeout is reasonable, as a longer latency period would reduce transmission by preventing individuals from becoming infectious. This would result in a smaller outbreak, which would have a greater probability of fading out.  

It seems reasonable that population level parameters like K, $\gamma$, and $\theta$ would be related with all metrics. Previous results demonstrated the importance of herd size on these metrics, but we suspect these mortality parameters are relevant because they regulate the turnover rate within the population. It seems reasonable that these would be related to infection prevalence and fadeout because the increased turnover rates facilitate clearing infected individuals from the population.

\end{comment}

\begin{comment}

\section{Cover Letter}

Drs. Doherr, Fosgate, and Kostoulas
Editors-in-Chief
Preventative Veterinary Medicine

June 24, 2022

Dear Drs. Doherr, Fosgate, and Kostoulas and the Preventative Veterinary Medicine Editorial Board,


Please find enclosed our manuscript entitled “Stochastic modeling of bovine tuberculosis dynamics in white-tailed deer” for your consideration as a research article in Preventive Veterinary Medicine.

Wildlife disease management can be difficult and costly but is often crucial in development of policy regarding livestock health and food safety. In the case of bovine tuberculosis (bTB), eradication in livestock has nearly been achieved in the United States through current management practices; however, white-tailed deer in Michigan have become established as a reservoir host of the disease. As this allows for near constant potential for spillover into livestock, there is interest in how to best mitigate these risks. Here, we use a stochastic modeling approach to simulate bTB dynamics in this wildlife system as an alternative to field-based research. We can use this mathematical model to predict disease outcomes over longer periods of time to inform the relative risk of bTB spillover to livestock populations over time. Such simulations also allow for exploration of different intervention and outbreak scenarios to understand what control measures are economically feasible and optimally effective. We find that while infected livestock can serve as the source of a bTB outbreak in deer populations, the long term trajectory is largely determined by intraspecific transmission, and the number of individuals that become infected at the beginning of the outbreak largely dictates whether the outbreak will lead to an endemic disease state in the population. 

We believe this study provides valuable insights to developing effective wildlife disease management policies; similarly, this is also applicable to livestock management policy making in regions where wildlife and livestock might overlap. Disease spillover at the wildlife-livestock interface occurs globally across different hosts and disease systems. The methods used in this work can be generalized to other wildlife disease systems of agricultural importance, so we believe this work would have a broad appeal to scientists and policymakers concerned with animal health. Because of the application to animal health in both wildlife and livestock, we believe that this study would be a good fit for publication in Preventive Veterinary Medicine.

This work has not been submitted elsewhere and has been approved for submission by all co-authors. The authors have no competing interests. USDA has also approved this manuscript for submission as required by those agencies based on co-authorship and funding.

Thank you for your consideration.

Sincerely,

Brandon J. Simony
\end{comment}

%Because infection fadeout is unlikely, measures to reduce the risk of spillback should be implemented. Removal of animals that frequently visit farms may be a useful first initiative, as ante mortem diagnostics for bTB in cattle are largely imperfect, resulting in potential cryptic infection that may lead to larger scale outbreaks among cattle. Loosening restrictions on farmers' ability to hunt wildlife on their property, increased fencing around pastures, and covered storage of feed and water troughs could all serve to reduce transmission potential between cattle and deer, which may lead to reduced prevalence in wildlife. Although it is a consistent underestimate, prevalence estimates from hunter harvest may still be useful, as year to year variation in detection could be used to justify changes in intervention intensity.


%\comment{Unlike on cattle farms, where the population is highly managed and surveillance is feasible at a population level, exact wildlife population prevalence is not readily available, and the same level of disease surveillance is fiscally infeasible. As such, direct disease management for bTB in wildlife is not economically feasible, so mathematical models are a useful tool in predicting the disease dynamics of bTB within a wildlife population. By implementing a model that attempts to capture the heterogeneous contact rates of a select few individuals, we may be able to better understand the importance of these individuals on wildlife disease dynamics as well as potential risk factors. By explicitly including these individuals within a stochastic modeling framework\citet{SANDMANN}, it may be possible to predict the effect of these `super spreaders' on the dynamics of bTB within a wildlife population when the only risk of infection is from fomite contact with farms (initially). Additionally, we can explore the risk to these `super spreading' individuals as a risk factor of spill back into livestock by introducing bTB infection to a herd and observing the number of these `super spreaders' become infected.\CTW{THIS IS A GOOD INTRO BUT IT DOESN"T INCLUDE THE SURVEILLANCE PIECE IN THE ABSTRACT. MAYBE NEED TO ADD THAT IN TO PROVIDE CONTEXT FOR THAT IDEA IN THE REST OF THE PAPER?}}